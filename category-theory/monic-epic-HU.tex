\documentclass{article}

\usepackage[utf8]{inputenc}
\usepackage{t1enc}
\usepackage[magyar]{babel}
\sloppy

\usepackage{proof}

\usepackage{amsmath}
\usepackage{amssymb}
\usepackage{amsthm}

\usepackage{BOONDOX-calo}
\newcommand{\obSet}{\ob_{\setCat}}
\newcommand{\ob}{\mathsf{Ob}}
\newcommand{\setCat}{\mathbf{Set}}

\newcommand{\parenth}[1]{\left(#1\right)}
\newcommand{\setOf}[1]{\left\lbrace\,#1\,\right\rbrace}
\newcommand{\setAbs}[2]{\left\lbrace\,\,#1\,\middle|\,\,#2\,\right\rbrace}

\usepackage{stmaryrd}

\newtheorem{thm}{tétel}

\usepackage{comment}

\begin{document}
	\[
		\infer[(\lor E)_{a,b}]{C}{
			A\lor B&
			%\infer[a]A{}&
			%\infer[b]B{}
			\infer*C{
				\infer[a]A{}
			}&
			\infer*C{
				\infer[b]B{}
			}
		}
	\]
	\begin{thm}
		Legyen $X, Y \in \obSet, e: X \to Y$. Ekkor
		\[
			\infer*{
				\forall y \in Y \bullet \exists x \in X \bullet ex = y
			}
			{
				\forall Z \in \obSet \bullet \forall f_1, f_2: Y \to Z \bullet e\ogreaterthan f_1 = e\ogreaterthan f_2 \to f_1 = f_2
			}
		\]
	\end{thm}
	\begin{proof}
		Elhagyva az intuicionista logika területét, megelégszünk azzal hogy az intuicionista logikán belül az alábbi tétel bizonyításával is megelégszünk:
	\end{proof}
	\begin{thm}
		\[
			\infer*{
				\exists Z \in \obSet \bullet \exists f_1, f_2: Y \to Z \bullet f_1 \neq f_2 \land e\ogreaterthan f_1 = e\ogreaterthan f_2
			}
			{
				\exists y_0 \in Y \bullet \forall x \in X \bullet ex \neq y_0
			}
		\]
	\end{thm}
	\begin{proof}
		Konzervatív bővítés (,,szintaktikus cukor''):
		\begin{align}
			\mathcal Z    &:\equiv \setOf{0, +1, -1},\\
			\mathcal{f_1} &:\equiv \setAbs{y \mapsto \begin{cases}+1&\text{ha }y=y_0\\0&\text{ha } y\neq y_0\end{cases}\in Y\times\mathcal Z}{y\in Y},\\
			\mathcal{f_2} &:\equiv \setAbs{y \mapsto \begin{cases}-1&\text{ha }y=y_0\\0&\text{ha } y\neq y_0\end{cases}\in Y\times\mathcal Z}{y\in Y}.
		\end{align}
		Ezzel a jelöléssel már:
		\[
			\infer[\parenth{\exists I}]{
				%\exists Z \in \obSet \bullet \exists f_1, f_2: Y \to Z \bullet f_1 \neq f_2 \land e\ogreaterthan f_1 = e\ogreaterthan f_2
				\exists f_1 \neq f_2 \bullet e\ogreaterthan f_1 = e\ogreaterthan f_2
			}
			{
				\infer[(\exists E)_a]{
					\mathcal{f_1} \neq \mathcal{f_2} \land e\ogreaterthan \mathcal{f_1} = e\ogreaterthan \mathcal{f_2}
				}{
					\exists y_0 \bullet \forall x \bullet ex \neq y_0
					&
					\infer[(\land I)]{
						\mathcal{f_1} \neq \mathcal{f_2} \land e\ogreaterthan \mathcal{f_1} = e\ogreaterthan \mathcal{f_2}
					}{
						\infer{
							\mathcal{f_1} \neq \mathcal{f_2}
						}{
							\infer[(\exists I)]{
								\exists y \bullet \mathcal{f_1}y \neq \mathcal{f_2}y
							}{
								\infer{
									\mathcal{f_1}y_0 \neq \mathcal{f_2}y_0
								}{
									\infer{\mathcal{f_1}y_0 = +1}{}
									&
									\infer{\mathcal{f_2}y_0 = -1}{}
								}
							}
						}
						&
						\infer{
							e\ogreaterthan \mathcal{f_1} = e\ogreaterthan \mathcal{f_2}
						}{
							\infer[(\forall I)]{
								%\forall x \in X \bullet \mathcal{f_1(ex)} = \mathcal{f_2(ex)}
								\forall x \bullet \mathcal{f_1(ex)} = \mathcal{f_2(ex)}
							}{
								\infer{
									\mathcal{f_1}(ex) = \mathcal{f_2}(ex)
								}{
									\infer{
										\mathcal{f_1}(ex) = 0
									}{
										\infer[(\forall E)]{
											ex \neq y_0
										}{
											\infer[a]{
												\forall x \bullet ex \neq y_0
											}{}
										}
									}
									&
									\infer{
										\mathcal{f_2}(ex) = 0
									}{
										\infer[(\forall E)]{
											ex \neq y_0
										}{
											\infer[a]{
												\forall x \bullet ex \neq y_0
											}{}
										}
									}
								}
							}
						}
					}
				}
			}
		\]%x \in X \to \mathcal{f_1}\parenth{ex} = 0 \land \mathcal{f_2}\parenth{ex} = 0
	\end{proof}
	\begin{comment}
		\infer[\parenth{\forall I}]{
			\forall x \in X \bullet \mathcal{f_1}\parenth{ex} = \mathcal{f_2}\parenth{ex}
		}{
			\infer[\parenth{\to I}_a]{
				x \in X \to \mathcal{f_1}\parenth{ex} = \mathcal{f_2}\parenth{ex}
			}{
				\infer{
					\mathcal{f_1}\parenth{ex} = \mathcal{f_2}\parenth{ex}
				}{
					\infer[a]{x \in X}{}
				}
			}
		}
	\end{comment}
\end{document}
